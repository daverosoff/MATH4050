\documentclass{article}
\usepackage{amsmath, amsthm, amssymb, mathtools, enumitem, xcolor}
\usepackage[margin=1in]{geometry}
\begin{document}
\title{Homework 1 Solutions}
\author{Your Name}
\maketitle
\section*{Problem 1}
\textbf{Part b.} Prove that for all $a$, $b$, $c \in \mathbf{N}$, we have
\[
    ac = bc \implies a = b.
\]
\begin{proof}
Assume $ac = bc$. We want to show that $a = b$. The proof is by the trichotomy
principle for natural numbers. We show that the other two cases, namely
\[
    a < b \quad \text{and} \quad a > b
\]
are impossible. To see this, assume for the sake of contradiction that $a < b$. Then we have
\[
    ac < bc,
\]
which contradicts our assumption that $ac = bc$. A similar argument shows that the case 
$a > b$ is also impossible. Therefore we must have $a = b$.
\end{proof}
\section*{Problem 2}
\textbf{Part a.} Prove that for all $n \in \mathbf{N}$, we have $n \ne n + 1$.
\begin{proof}
    The proof is by induction on $n$. The base case is $n = 1$. We need to prove
    \[
        1 \ne 2,
    \]
    but because $2 = s(1)$, this is true by the Peano axioms which say that 
    $1$ is not the successor of any natural number. For the inductive step, assume
    that for some $k \in \mathbf{N}$, we have
    \[
        k \ne k + 1.
    \]
    Because $s$ is injective, we have
    \[
        s(k) \ne s(k + 1).
    \]
    This concludes the proof.
\end{proof}
\textbf{Part b.} Prove that for all $n, k\in \mathbf{N}$, we have
\[
    n \ne n + k.
\]
\begin{proof}
    The proof is by induction on $k$. The base case is $k = 1$, which is exactly what we
    proved in part (a). For the inductive step, assume that for some $k \in \mathbf{N}$,
    we have
    \[
        n \ne n + k,
    \]
    for all $n \in \mathbf{N}$. We need to show that for all $n \in \mathbf{N}$,
    \[
        n \ne n + s(k).
    \]
    If $n = 1$, then $n + s(k) = 1 + s(k) = s(1 + k)$, which is not equal to $1$
    because, by the Peano axioms, $1$ is not the successor of any natural number.

    Now suppose we have shown that for some $n \in \mathbf{N}$,
    \[
        n \ne n + s(k).
    \]
    Because $s$ is injective, we have
    \[
        s(n) \ne s(n + s(k)) = s(n) + s(k).
    \]
    This concludes the proof.
\end{proof}
\section*{Problem 3}
\textbf{Part a.} Prove that, for all $x$, $y \in \mathbf{Z}$, we have
\[
    x + y = y + x.
\]
\begin{proof}
    Let $x$ and $y$ be represented by the pairs $(a, b)$ and $(c, d)$ respectively, where
    $a$, $b$, $c$, $d \in \mathbf{N}$. Then we have
    \[
        x + y = (a, b) + (c, d) = (a + c, b + d).
    \]
    By the commutativity of addition in $\mathbf{N}$, we have
    \[
        (a + c, b + d) = (c + a, d + b).
    \]
    But
    \[
        (c + a, d + b) = (c, d) + (a, b) = y + x.
    \]
    This concludes the proof.
\end{proof}
\textbf{Part b.} Prove that, for all $x$, $y \in \mathbf{Z}$, we have
\[
    x + \overline{0} = x.
\]
\begin{proof}
    Let $x$ be represented by the pair $(a, b)$, where $a$, $b \in \mathbf{N}$. Then we have
    \begin{align*}
        x + \overline{0} &= [(a, b)] + [(0, 1)] = [(a \cdot 1 + b \cdot 0, b \cdot 1)] \\
        &= [(a, b)] = x.
    \end{align*}
    This concludes the proof.
\end{proof}
\section*{Problem 4}
Show that multiplication on $\mathbf{Z}$, defined by
\[
    [(a, b)] \cdot [(c, d)] = [(ac + bd, ad + bc)],
\]
is well-defined.
\begin{proof}
    Let $[(a, b)] = [(a', b')]$ and $[(c, d)] = [(c', d')]$. We need to show that
    \[
        [(a, b)] \cdot [(c, d)] = [(a', b')] \cdot [(c', d')].
    \]
    By the definition of equality in $\mathbf{Z}$, we have
    \[
        a + b' = a' + b \quad \text{and} \quad c + d' = c' + d.
    \]
    We need to show that
    \[
        [(ac + bd, ad + bc)] = [(a'c' + b'd', a'd' + b'c')].
    \]
    By the definition of equality in $\mathbf{Z}$, this is equivalent to showing that
    \[
        (ac + bd) + (a'd' + b'c') = (a'c' + b'd') + (ad + bc).
    \]
    Rearranging terms, this is equivalent to showing that
    \[
        ac + b'c' + a'd' + bd = a'c' + ad + bc + b'd'.
    \]

    Using the two equalities we have, $a + b' = a' + b$ and $c + d' = c' + d$, 
    we can add these equalities in various combinations to express all terms 
    using only addition. Expanding both sides of the equation and grouping terms 
    appropriately, we see that both sides are equal by the commutativity and associativity of addition in $\mathbf{N}$. Thus, the multiplication is well-defined.
\end{proof}
\section*{Problem 5}
For $[(a, b)] \in \mathbf{Z}$, show that $[(a, b)] = \overline{0}$ if and only if $a = b$.
\begin{proof}
    By the definition of equality in $\mathbf{Z}$, we have
    \[
        [(a, b)] = \overline{0} \iff (a, b) \sim (0, 1) \iff a + 1 = b + 0 \iff a = b.
        \]
        This concludes the proof.
    \end{proof} 
\section*{Problem 6}
For $[(a, b)] \in \mathbf{Z}$, show that $\overline{0} < [(a, b)]$ if and only if $b < a$.
\begin{proof}
    By the definition of the order relation in $\mathbf{Z}$, we have
    \[
        \overline{0} < [(a, b)] \iff (0, 1) < (a, b) \iff 0 + b < 1 + a \iff b < a.
    \]
    This concludes the proof.
\end{proof}
\section*{Problem 7}
\textbf{Part a.} For $x$, $y$, $z \in \mathbf{Z}$, prove that if $xy = \overline{0}$
then $x = \overline{0}$ or $y = \overline{0}$.
\begin{proof}
    Suppose $x \ne \overline{0}$. Then either $x > \overline{0}$ or $x < \overline{0}$.
    If $x > \overline{0}$, then $x$ is represented by
    \[
        (a, b) \quad \text{where } a, b \in \mathbf{N} \text{ and } a > b.
    \]
    If $x < \overline{0}$, then $x$ is represented by
    \[
        (a, b) \quad \text{where } a, b \in \mathbf{N} \text{ and } a < b.
    \]
    In either case, we have $a \ne b$. Now let $y$ be represented by $(c, d)$,
    where $c$, $d \in \mathbf{N}$. We have
    \[
        xy = [(a, b)] \cdot [(c, d)] = [(ad + bc, bd)].
    \]
    We want to show that $y = \overline{0}$, which is equivalent to showing that
    \[
        (ad + bc, bd) \sim (0, 1) \iff ad + bc + 1 = bd + 0 \iff ad + bc + 1 = bd.
    \]
    Rearranging terms, this is equivalent to showing that
    \[
        ad + bc + 1 - bd = 0 \iff ad + bc + 1 = bd.
    \]
    This is a linear equation in $c$ and $d$. Because $a \ne b$, we can solve for $c$ in terms of $d$ (or vice versa). Thus, there exists a unique solution for $c$ and $d$ in $\mathbf{N
        
\end{document}